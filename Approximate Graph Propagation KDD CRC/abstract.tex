\begin{abstract}
Efficient computation of node proximity queries such as transition probabilities, Personalized PageRank, and Katz are of fundamental importance in various graph mining and learning tasks. In particular, several recent works leverage fast node proximity computation to improve the scalability of Graph Neural Networks (GNN). However, prior studies on proximity computation and GNN feature propagation are on a case-by-case basis, with each paper focusing on a particular proximity measure. % Moreover, existing feature propagation methods in GNNs rarely leverage approximate algorithms for proximity computation, limiting their scalability on billion-scale graphs. 
%Efficient computation of node proximity queries such as transition probabilities, Personalized PageRank, and Katz are of fundamental importance in various graph mining and learning tasks, including graph clustering, Graph Neural Network(GNN), and label propagation. However, prior studies on proximity computation are on a case-by-case basis, with each paper focusing on a particular proximity measure. Moreover, several recent works leverage fast node proximity computation to improve scalability. In particular, existing feature propagation methods in GNNs rarely leverage approximate algorithms for proximity computation, limiting their scalability on billion-scale graphs. 

In this paper, we propose {Approximate Graph Propagation (AGP)}, a unified randomized algorithm that computes various proximity queries and GNN feature propagations, including transition probabilities, Personalized PageRank, heat kernel PageRank, Katz, SGC, GDC, and APPNP. Our algorithm provides a theoretical bounded error guarantee and runs in almost optimal time complexity. We conduct an extensive experimental study to demonstrate AGP's effectiveness in two concrete applications: local clustering with heat kernel PageRank and node classification with GNNs. Most notably, we present an empirical study on a  billion-edge graph Papers100M, the largest publicly available  GNN dataset so far. The results show that AGP can significantly improve various existing GNN models' scalability without sacrificing prediction accuracy. %Codes for our experiments and the technical report with detailed proofs can be found at~\cite{TechnicalReport}. 

%In this paper, we propose {\it Approximate Graph Propagation (AGP)}, a unified randomized algorithm that computes various proximity queries. Our algorithm provides a theoretical bounded error guarantee and runs in almost optimal time complexity. Besides, we conduct an extensive experimental study to demonstrate AGP's effectiveness in two concrete applications: local clustering and node classification with GNNs. Most notably, we present an empirical study on a billion-edge graph Papers100M, the largest publicly available GNN dataset so far. The results show that AGP can significantly improve various existing GNN models' scalability without sacrificing prediction accuracy. 


\end{abstract}

%%% Local Variables:
%%% mode: latex
%%% TeX-master: "paper"
%%% End:
